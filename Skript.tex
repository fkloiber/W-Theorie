\documentclass[parskip = full, 12pt]{scrartcl}

%%% Packages
  \usepackage{ucs}
  \usepackage[utf8x]{inputenc}
  \usepackage[T1]{fontenc}
  \usepackage[ngerman]{babel}
  \usepackage{amsmath}
  \usepackage{amssymb}
  \usepackage{amsfonts}
  \usepackage{lmodern}
  \usepackage{enumitem}
  \usepackage{bm}
  \usepackage{xcolor}
  \usepackage{sectsty}
  \usepackage{tikz}
  \usepackage{caption}
  \usepackage{array}
  \usepackage{tabularx}
  \usepackage{tabulary}

%%% Settings
  \setenumerate[1]{label={\alph*)}}
  \setenumerate[0]{label={\roman*)}}
  \usetikzlibrary{arrows,decorations.pathmorphing,backgrounds,positioning,
    fit,petri}

%%% Definitions
  \definecolor{Green}{rgb}{0.0, 0.6, 0.0}
  \definecolor{Blue}{rgb}{0.0, 0.0, 0.6}
  \definecolor{Red}{rgb}{0.6, 0.0, 0.0}
  \definecolor{Orange}{rgb}{0.9, 0.5, 0.1}
  \definecolor{Violet}{rgb}{0.6, 0.2, 0.8}


%%% Commands

% Disjoint union and intersection
  \makeatletter
  \def\moverlay{\mathpalette\mov@rlay}
  \def\mov@rlay#1#2{\leavevmode\vtop{%
     \baselineskip\z@skip \lineskiplimit-\maxdimen
     \ialign{\hfil$\m@th#1##$\hfil\cr#2\crcr}}}
  \newcommand{\charfusion}[3][\mathord]{
      #1{\ifx#1\mathop\vphantom{#2}\fi
          \mathpalette\mov@rlay{#2\cr#3}
        }
      \ifx#1\mathop\expandafter\displaylimits\fi}
  \makeatother

  \newcommand{\cupdot}{\charfusion[\mathbin]{\cup}{\cdot}}
  \newcommand{\bigcupdot}{\charfusion[\mathop]{\bigcup}{\cdot}}
  \newcommand{\capdot}{\charfusion[\mathbin]{\cap}{\cdot}}
  \newcommand{\bigcapdot}{\charfusion[\mathop]{\bigcap}{\cdot}}

% Powerset and other setnames
  \newcommand{\Pot}{\mathcal{P}}
  \newcommand{\Bor}{\mathcal{B}}
  \newcommand{\sD}{\mathcal{D}}
  \newcommand{\sA}{\mathcal{A}}
  \newcommand{\sE}{\mathcal{E}}
  \newcommand{\sR}{\mathcal{R}}
  \newcommand{\sS}{\mathcal{S}}
  \newcommand{\sM}{\mathcal{M}}
  \newcommand{\sF}{\mathcal{F}}
  \newcommand{\Om}{\Omega}
  \newcommand{\om}{\omega}

% real and natural numbers
  \newcommand{\Nat}{\mathbb{N}}
  \newcommand{\Real}{\mathbb{R}}
  \newcommand{\Com}{\mathbb{C}}

% Varianz, Kovarianz and Expected Value
  \newcommand{\Var}{\textrm{Var}}
  \newcommand{\E}{\textrm{E}}
  \newcommand{\Cov}{\textrm{Cov}}

%coloured subsections
  \newcommand{\colorsubsection}[2][black]{
      \subsectionfont{\color{#1}}
      \subsection{#2}
      \subsectionfont{\color{black}}
  }

% Text over symbols
  \newcommand{\symtext}[2]{
    \mathop{#1} \limits^{\mathop{\downarrow} \limits^{\textrm{#2}}}
  }


\title{Einführung in die Wahrscheinlichkeitstheorie und Statistik}
\author{Prof. Dr. Niko Naumann}
\date{\today{}}

\begin{document}

\maketitle
\tableofcontents

\section{Erinnerungen an die Maßtheorie} 
\label{s: 1.Erinnerungen an die Masstheorie}

\colorsubsection[Blue]{Defintion}
\label{ss: Definition 1.1}
	Sei $\Om$ eine Menge, $\Pot(\Om) := \{A\ |\ A \subseteq \Om\}$ die Potenzmenge 
	von $\Om$ auf $\sA \in \Pot(\Om)$.
	\begin{enumerate}[label={\alph*)}]
		\item $\sA$ heißt $\bm{\sigma}$\textbf{-Algebra} (auf $\Om$), falls gelten:
			\begin{enumerate}[label={\roman*)}]
				\item $\emptyset \in \sA$
				\item $\forall A \in \sA$ gilt $A^C := \{ \om \in \Om\ |\ \om \notin A\} 
				  \in \sA$
				\item $\forall A_n \in \sA (n \in \Nat)$ gilt $\bigcup_n A_n \in 
				  \sA$
			\end{enumerate}
			Ein \textbf{Messraum} ist ein Paar $(\Om, \sA)^{n \in \Nat}$ bestehend aus 
			einer Menge $\Om$ und einer $\sA \in \Pot(\Om)$.
		\item Falls in a) statt iii) nur gilt:
			\begin{enumerate}[label={\roman**)}] \setcounter{enumii}{2}
				\item Für $A,B \in \sA$ gilt $A \cup B \in \sA$ 
			\end{enumerate}
			so heißt $\sA$ eine \textbf{Algebra} (auf $\Om$)
		\item Falls in a) statt iii) nur gilt:
			\begin{enumerate}[label={\roman***)}]
			\setcounter{enumii}{2}
				\item Falls $A_n \in \sA (n \in \Nat)$ paarweise disjunkt sind (d.h. 
				$A_m \cap A_n \neq \emptyset$ für $n \neq m$), dann ist $\bigcupdot
				_{n \in \Nat} A_n \in \sA$
			\end{enumerate}
			so heißt $\sA$ ein \textbf{Dynkin System}.
	\end{enumerate}

\colorsubsection[Blue]{Bemerkung \& Definition}
\label{ss: Bemerkung und Definition 1.2}
	\begin{enumerate}[label={\alph*)}]
		\item Die größte $\sigma$-Algebra auf $\Om$ ist $\Pot(\Om)$, die kleinste 
			ist $\{\emptyset,\Om\}$. Falls $I \neq \emptyset$ eine Menge und für jedes 
		  $i \in I\ \sA_i \subseteq \Pot(\Om)$ eine $\sigma$-Algebra ist, so auch 
		  $\bigcap_{i \in \Nat}	\sA_i \subseteq \Pot(\Om)$.\\
			Sei $\sE \subseteq \Pot(\Om)$ beliebig, dann ist $$\sigma(\sE) := \bigcap 
			\{ \sA\ |\ \sA \subseteq \Pot(\Om) \textrm{ ist eine } \sigma \textrm{
			-Algebra mit } \sE \subseteq \sA \} \subseteq \Pot \Om$$ die kleinste 
			$\sigma$-Algebra, die $\sE$ enthält (\textbf{die von $\sE$ erzeugte 
			$\sigma$-Algebra)}. Analog existieren ein kleinstes Dynkin-System 
			$\sD(\sE)$ und eine kleinste Algebra, die $\sE$ die enthalten.\\
			Die von $\sE$ erzeugte Algebra kann man angeben, nämlich $\{\bigcupdot
			_{i=1}^n \bigcapdot_{j=1}^n A_{ij}\ | \ A_{ij} \in \sE \cup	
			\sE^C ,n \geq I \}$, wobei $\sE^C := \{ A^C \ |\ A \in \sE$. Für $\sigma$
			-Algebren gilt nichts analoges.
		\item Ein \textbf{Ring} (auf $\Om$) ist eine Teilmenge $\sA \subseteq 
		  \Pot(\Om)$ mit i) iii*) und ii*) Für alle $A, B \in \sA$ ist $A\backslash 
		  B \in \sA$.  
	\end{enumerate}

\colorsubsection[Orange]{Lemma}
\label{ss: Lemma 1.3}
	\begin{enumerate}[label = {\alph*)}]
	  \item Ein $\cap$-stabiles Dynkin-System ist eine $\sigma$-Algebra.
	  \item \textbf{(Dynkin-Lemma)} Sei $\sE \in \Pot(\Omega) \ \cap$-stabil,  
	    dann ist $\sD(\sE)  = \sigma(\sE)$ 
	\end{enumerate}
	\textbf{Beweis}: Kommt später!

\colorsubsection[Blue]{Definition}
\label{ss: Definition 1.4}
	\begin{enumerate}
		\item Sei $(\Om, \sA)$ ein Messraum. Dann heißt eine Abbildung $\mu: \sA 
			\rightarrow [0,\infty] := [0,\infty) \cupdot \{\infty\}$ ein 
			\textbf{Maß} (auf ($\Om, \sA$), falls gelten:
			\begin{enumerate}
				\item $\mu(\emptyset) = 0$
				\item $\sigma$-Additivität: Für alle $(\sA)_{n \geq 1} \subseteq 
					\sA$ mit $A_n \cap A_m = \emptyset$ für $n \neq m$ gilt $\mu 
					( \bigcupdot_{n \geq 1} A_n) = \sum_{n \geq 1} \mu(A_n)$
			\end{enumerate}
		\item Ein Maß $\mu$ heißt 
			\begin{enumerate}
				\item \textbf{endlich} $:\Leftrightarrow \mu(\Om) \leq \infty$
				\item $\bm{\sigma}$\textbf{-finit} $:\Leftrightarrow$ Es exisitiert 
					eine Folge ${(A_n)}_{n \geq 1} \subseteq \sA$ mit $\mu(A_n) \leq 
					 \infty\ \forall n \leq 1$ und $\bigcup_{n \geq 1} A_n = 
					 \Om$
				\item \textbf{Wahrscheinlichkeitsmaß} (W-Maß), falls gilt $\mu(\Om)
					= 1$. 
			\end{enumerate}
		\item Sei $\sA$ ein Ring auf $\Om$, dann heißt eine Abbildung $\mu: \sA 
			\rightarrow [0, \infty]$ ein \textbf{Inhalt}, falls gelten:
			\begin{enumerate}
				\item $\mu(\emptyset) = 0$
				\item $\mu(A \cup B) = \mu(A) + \mu(B)\ \forall A,B \in \sA$ mit $A 
				\cap B = \emptyset$ \textbf{(endliche Additivität)}.
			\end{enumerate}
	\end{enumerate}

\colorsubsection[Green]{Beispiel}
\label{ss: Beispeil 1.5}
	\begin{enumerate}
		\item \textbf{Dirac-Maß}: Zu $x \in \Om$ definiere für alle $A \in \Pot(\Om)
			:\delta_x(A) := \left \{ 
				\begin{tabular}{cl}
				1, & $x \in A$\\
				0, & $x \notin A$
				\end{tabular}
			\right. $. Dann ist ein $\delta_x$ ein Maß auf $\Pot(\Om)$ (und damit 
			auf jeder $\sigma$-Algebra auf $\Om$).
		\item \textbf{Elementargeometrischer Inhalt}: Betrachte $\sE := \{ (a,b]\ 
			|\ \leq a \leq b \leq +\infty \} \subseteq \Pot(\Om := \Real)$ und $\sR 
			:= \{ \bigcupdot_{i=1}^n A_i\ |\ A_i \in \sE, A_i \cap A_j \neq 
			\emptyset$ für $i \neq j, n \geq 1\} \subseteq \Pot(\Real)$. Dann ist 
			$\sR$ ein Ring. Setzte $\lambda(\bigcup_{i=1}^n A_i := \sum
			_{i = 1}^n (b_i - a_i)$ für $A_i = (a_i, b_i] \in \sE$ paaarweise
			disjunkt. Dann ist $\lambda: \Real \rightarrow [0, \infty)$ ein Inhalt.
		\item Sei $\Om = \Nat, \sA := \{A \subseteq \Nat\ |\ A$ oder $A^C$ ist 
			endlich \} und $$\mu(\sA) := \left \{
				\begin{tabular}{cl}
				0, & $A$ endlich \\
				1, & $A^C$ endlich
				\end{tabular}
			\right. {\forall A \in \sA}$$. Dann ist $\sA$ eine Algebra auf $\Nat$. 
			Und $\mu$ ist ein Inhalt auf $\sA$, aber $\sA$ ist keine 
			$\sigma$-Algebra (und $\mu$ ist nicht $\sigma$-additiv).
		\item \textbf{Zählmaß}: Für $A \subseteq \Om$ definiere $\mu(A) := \left \{
				\begin{tabular}{cl}
				$|A|$, 		& falls $A \leq \infty$ \\
				$\infty$,	& sonst
				\end{tabular}
			\right.$. Dann ist $\mu$ ein Maß auf $\Pot(\Om)$, welches genau dann 
			$\sigma$-finit ist, wenn $\Om$ abzählbat ist.
	\end{enumerate}

\colorsubsection[Violet]{Bemerkung}
\label{ss:  Bemerkung 1.6}
	Sei $\mu$ ein Inhalt auf einem Ring $\sA$. Dann gelten:
	\begin{enumerate}
		\item $\mu$ ist \textbf{monoton}, d.h. für $A,B \in \sA$ mit $A \subseteq B$
			gilt $\mu(A) \leq \mu(B)$
		\item $\mu$ ist \textbf{subtraktiv}, d.h. für $A,B \in \sA$ mit $A \in B$ 
			und	$\mu(A) \leq \infty$ gilt $\mu(B \backslash A) = \mu(B) - \mu(A)$
		\item $\mu$ ist \textbf{subadditiv}, d.h. für $A_1,...,A_n \in \sA$ gilt
			$\mu (\bigcup_{i=1}^n A_i) \leq \sum_{i=1}^n \mu(A_i)$
	\end{enumerate}

\colorsubsection[Red]{Satz}
\label{ss: Satz 1.7}
	Sei $\mu$ ein Inhalt auf einem Ring $\sA$. Betrachte folgende Aussagen:
	\begin{enumerate}
		\item $\mu$ ist $\sigma$-additiv
		\item Für alle $\{A_1,A_2,...\} \subseteq \sA$ mit $A_1 \subseteq A_2 
			\subseteq ...$ und $A := \bigcup_{n \geq 1} A_n \in \sA$ gilt 
			$\mu(A) =\lim_{n \rightarrow \infty} \mu(A_n)$ (d.h. $\mu$ 
			ist \textbf{stetig von unten}). 
		\item Für alle $\{A_1,A_2,...\} \subseteq \sA$ mit $A_1 \supseteq A_2 
			\supseteq ..., \bigcap_{n \geq 1} A_n = \emptyset$ und $\mu(A_1)
			\leq \infty$ gilt $\lim_{n \rightarrow \infty} \mu(A_n) = 0$ 
			(d.h.) $\mu$ ist \textbf{stetig von oben}. Dann gilt $a) \Leftrightarrow
			b) \Rightarrow c)$. Falls $\mu$ endlich ist, so gilt außerdem $c)
			\Leftrightarrow b)$
	\end{enumerate}
	\textbf{Beweis:} Kommt später!

\colorsubsection[Green]{Definition}
\label{ss: Defintion 1.8}
	Seien $\Om$ eine Menge, $\mu$ ein $\sigma$-additiver Inhalt auf einem Ring
	$\sA \subseteq \Pot(\Om)$ und $B \subseteq \Om$
	\begin{enumerate}
		\item Es heißt $\mu^* : \Pot(\Om) \rightarrow [0, \infty]$, definiert durch
			$$ \mu^*(B) := \left\{ 
				\begin{tabular}{ccl}
					$ \inf \{ \sum_{n \geq 1} \mu(A_n)\ |\ A_n \in \sA, B 
					\subseteq \bigcup_{n \geq 1} A_n\} $ &,& falls $\{...\} 
					\neq \emptyset$ \\
					$\infty$ &,& sonst
				\end{tabular}
			\right. $$
			\textbf{das dazugehörige Maß}.
		\item Eine Menge $A \subseteq \Om$ heißt \textbf{maßlos}, falls gilt $\mu^*
			(A) = \mu^*(A \cap B) + \mu^*(A \cap B^C)$ für alle $B \in \Pot(\Om)$
	\end{enumerate}

\colorsubsection[Red]{Satz}
\label{ss: Satz 1.9}
	\textbf{(Caratheodory)} Seien $\Om$ eine Menge und $\mu$ ein $\sigma$-additiver, 
	$\sigma$-finiter Inhalt auf einem Ring $\sA$ auf $\Om$. Dann ist $\sigma(\sA)
	\subseteq \overline{\sigma}(\sA) := \{ B \subseteq \Om\ |\ B $ messbar $\}$ 
	eine $\sigma$-Algebra und $\mu^*|_{\overline{\sigma}(\sA)}$ ist ein Maß. Das 
  Maß $\mu^*|_{\overline{\sigma}(\sA)}$ ist die einzige Maßfortsezung von $\mu$. 
	(ohne Beweis) \\
	Die $\sigma$-Algebra $\overline{\sigma}$ hat eine Bedeutung weiter unten.
	(siehe Satz 1.11)

\colorsubsection[Blue]{Definition}
\label{ss: Definition 1.10}
	Ein Maß $\mu$ auf einer $\sigma$-Algebra $\sA$ heoßt \textbf{vollständig}, 
	falls gilt: Aus $A \subseteq B  \subseteq \Om, B \subseteq \sA$ und 
	$\mu(B) = 0$ folgt $A \subseteq \sA$ (und natürlich $\mu(A) = 0$). Ein 
	vollständiges Maß $\mu: A	\rightarrow [0, \infty]$ heißt \textbf{
	Vervollständigung des Maßes} $\bm{\mu_0: \sA \rightarrow [0;\infty]}$, falls 
	$\sA_0 \subseteq \sA,\ \mu|_{\sA_0} = \mu_0$ und folgende Eigenschaft gilt: 
	Sei $\mu^*:\sA^* \rightarrow [0, \infty]$ eine vollständige Fortsetzung 
	von $\mu_0$. Dann gelten $\sA \subseteq \sA^*$ und $\mu^*|_{\sA} = \mu$.\\
	\textbf{Definiton:} In der Situation von 1.8) setze $d_{\mu^*}(A,B) := \mu^*
	(A \Delta B)$ für alle $A, B \subseteq \Om$, wobei $A\Delta B := (A \backslash 
	B) \cup (B \backslash A)$ die \textbf{symmetrische Differenz von A und B} ist.

\colorsubsection[Red]{Satz} 
\label{ss: Satz 1.11}   
	Seien $\Om$ eine Menge, $\sA \subseteq \Pot(\Om)$ ein Ring und $\mu$ ein 
	$\sigma$-finiter und $\sigma$-additiver Inhalt auf $\sA$. Dann gelten:
	\begin{enumerate}
		\item $\mu^*|_{\overline{\delta}(\sA)}$ ist die Vervollständigung von 
			$\mu^*|_{\delta(\sA)}$
		\item $ \overline{\delta}(\sA) := \{ B \in \Pot(\Om)\ |\ \forall \epsilon 
			\geq 0\ \exists A \in \sA $ mit $d_{\mu^*}(A,B) \leq \epsilon \}$ 
			(ohne Beweis)
	\end{enumerate}			

\colorsubsection[Violet]{Bemerkung}
\label{ss: Bemerkung 1.12}
	\begin{enumerate}
		\item Die obigen Sätze besagen, dass ein $\sigma$-finiter und {$\sigma$
			-additiver} Inhalt auf einem Ring eindeutig ein Maß auf der erzeugten 
			$\sigma$-Algebra definiert. Das zum elementargeometrischen Inhalt 
			(Beispiel 1.5.c) gehörige Maß heißt \textbf{Lebesgue-Maß} $\lambda$, die 
			$\sigma$-Algebra ist die \textbf{Borel-Algebra} $\Bor(\Real)$. Die zur 
			Vervollständigung gehörende $\sigma$-Algebra heißt \textbf{das System der
			Lebesgue-messbaren Teilmengen auf $\Real$}.
		\item Es gilt zB.: $\overline{\sigma}(\sA) = \{ A \Delta N\ |\ A \in \sigma
			(\sA), N \subseteq \tilde{N} \in \sigma(\sA)$ mit $\mu(\tilde{N}) = 0 \}$.
			Für das Lebesgue-Maß gilt: Die Kardinalität von $\Bor(\Real)$ ist die von
			$\Real$, aber die Kardinalität der Lebesgue-messbaren Mengen ist 
			(mindestens) $\lambda^{|\Real|}$ und damit echt größer, denn: es genügt,
			eine Lebesgue-messbare Teilmenge $C \subseteq \Real$ zu finden, die 
			überabzählbar ist (also |C| = $|Real|$) und für die $lambda = 0$ gilt. 
			(Dann folgt $\Pot(C) \subseteq$ Lebesgue-messbar). Die Cantormenge C hat 
			diese Eigenschaft.
	\end{enumerate}

\colorsubsection[Blue]{Definition}
\label{ss: Definiton 1.13}
	Seien $(\Om, \sA)$ und $(S,\sS)$ Maßräume. Für eine Abbildung $X:\Om 
	\rightarrow S$ definiere für $B \subseteq S $:
	$$ X^{-1}(B) := \{ X \in B \} := \{ \om \in \Om\ |\ X(\om) \in B \} $$
	Dann heißt X $\sA-\sS$-messbar, falls für alle $B \in \sS$ gilt $X^{-1}(B)
	\in \sA$, d.h. es gilt \mbox{$X^{-1}(\sS) \in \sA$.}

\colorsubsection[Violet]{Bemerkung}
\label{ss: Bemerkung 1.14}
	\begin{enumerate}
		\item Jede konstante Abbildung is messbar bzgl. jeder $\sigma$-Algebra $\sA$
			(denn $X^{-1}(B) \in \{ \emptyset, \Om \} \in \sA$).
		\item Sind $X:(\Om, \sA) \rightarrow (S_1, \sS_1)$ und $Y:(S_1,\sS_1)
			\rightarrow (S_2, \sS_2)$ messbar, so auch $Y~\circ~X (\Om, \sA) 
			\rightarrow (S_2, \sS_2)$, denn $(Y \circ X)^{-1}(\sS_2) = X^{-1}(Y^{-1}
			(\sS_2)) \symtext{\subseteq}{Y messbar} X^{-1}(\sS_1) \symtext{\subseteq}
			{X messbar} \sA$
	\end{enumerate}

\colorsubsection[Orange]{Lemma}
\label{ss: Lemma 1.15}
  (Prüfen der Messbarkeit auf einem Erzeugendensystem). Seien $(\Om, \sA)$ und 
  $(S,\sS)$ Maßräume auf $\sS = \sigma(\sE)$ für ein $\sE \subseteq \Pot(\sS)$ 
  Dann ist eine Abbildung $X:\Om \rightarrow S$ genau dann $\sA-\sS-$messbar, 
  wenn $X^{-1}(\sE) \subseteq \sA$ gilt.

  \textbf{Beweis:} Kommt später!

  \textbf{Beispiel:} Falls in 1.15 $(S,\sS) = (\Real, \Bor(\Real))$ oder $(S, 
  \sS) = (\overline{\Real}, \overline{\Bor}(\Real))$ gilt, so folgt: Eine 
   Abbildung $X:(\Om, \sA) \rightarrow (S,\sS)$ ist genau dann messbar wenn für 
   alle $\alpha \in \Real$ gilt $\{ X \leq \alpha \} \in \sA = X^{-1}((-\infty, 
   \alpha])$. (Beachte $\Bor (\Real) = \sigma(\{ (-\infty, \alpha]\ |\ \alpha 
   \in \Real \})$ )

\section{Zufallsexperiment und Wahrscheinlichkeit}
\label{s: 2.Zufallsexperiment und Wahrscheinlichkeit}

\colorsubsection[Blue]{Definition}
\label{ss: Definiton 2.1}
	Ein \textbf{Zufallsexperiment} ist ein W-Raum $(\Om, \sA, P)$, d.h. ein
	Maßraum mit $P(\Om) = 1$. Ḿan sagt:
	\begin{enumerate}
	 	\item Ein $\om \in \Om$ heißt \textbf{Ergebnis} des Experiments.
	 	\item Ein $A \in \sA$ heißt \textbf{Ereignis} (die Wahl von $A$ ergibt
	 		sich aus dem Experiment).
	 	\item Für jedes $A \in \sA$ heißt $P(A)$ \textbf{die Wahrscheinlichkeit 
	 		 für das Eintreten des Ereignisses} $A \in \sA$. Das Maß $P:\sA 
	 		 \rightarrow [0,1]$	heißt die zum Experiment gehörige \textbf{
	 		 Wahrscheinlichkeitsverteilung (W-Verteilung)}. 
	 \end{enumerate} 

\colorsubsection[Violet]{Bemerkung}
\label{ss: Bemerkung 2.2}
	\begin{enumerate}
		\item Die Forderung, dass $\sA$ ein Algebra ist, ist naheliegend. Dabei 
			entspricht $\emptyset$ dem unmmöglichen Ereignis, $A \cap B$ dem 
			gleichzeitigen Eintreten von $A$ und $B$ und $A^C$ der logischen Negation
			von A. Die Forderung, dass $P$ ein Inhalt ist, ist naheliegend wegen der 
			Häufigkeitsinterpretation von Wahtscheinlichkeit. Die Bedingung, dass $\sA$
			ein $\sigma$-Algebra und P ein Maß sind, ist eine mathematische 
			Idealisierung (vgl. Satz 1.7, der dies als eine Stetigkeitseigenschaft 
			charakterisiert).
			\item Warum betrachtet man überhaupt $\sigma$-Algebren $\sA \subsetneqq
			\Pot(\Om)$? Ein Satz von Ulan besagt, dass es kein W-Maß P auf $\Pot([0,1]
			)$ mit $P\{x\} = 0$ für alle $x \in [0,1]$ gibt. Falls $\Om$ abzählbar ist
			(insbesondere zB. endlich) ist, wird man immer $\sA = \Pot(\Om)$ wählen. 
	\end{enumerate}

\colorsubsection[Green]{Beispiel}
\label{ss: Beispiel 2.3}
	\begin{enumerate}
		\item \textbf{Laplace-Experiment:} Hier ist $|\Om| < \infty$, d.h. wir 
			betrachten ein Experiment mit nur endlich vielen möglichen Ergebnissen,
			$\sA = \Pot(\Om\cdot)$ und $P(A) = \frac{|A|}{|\Om|}$ für alle $A \in \Pot(\Om)
			$. (Äquivalent: $\forall x \in \Om: P( \{x\} ) = \frac{1}{|\Om|}$.)
		\item \textbf{Gleichverteilung auf dem Intervall $\bm{[a,b]}\ (-\infty < a
			<	b < \infty)$:} Hier ist $\Om = [a,b],\ \sA = \Bor(\Om)$ und $P(A) = \frac
			{\lambda(A)}{\lambda(\Om)}$ für alle $A \in \sA$.
		\item \textbf{Mehrstufige Experimente:} Hier wird ein Zufallsexperiment
			$(\Om_1, \sA_1, P_1)$ n-fach wiederholt. Man erhält $\Om = \Om_1 \times 
			... \times \Om_1 = (\Om_1)^n,\ \sA = \sA_1 \otimes ... \otimes \sA_1$ und
			für das zugehörige Maß P gilt, falls $|\Om_1| < \infty$, die Darstellung
			$P(\{x_1, ..., x_n\}) = P_1(\{x_1\}) \cdot P_1(x_1; \{x_2\}) \cdot ...
			\cdot P_1(x_1, ..., x_{n-1};\{x_n\})$ wobei $P(x_1, ..., x_k; -)$  die 
			Übergangswahrscheinlichkeit für die $k$-te Wiederholung ist (Abhängig von
			$x_1, ..., x_k)$).

			\textbf{Beispiel:} Ziehen \textbf{ohne} Zurücklegen von zwei Kugeln aus
				einer Urne mit zwei schwarzen und einer weißen Kugel. Hier ist $n = 2,\ 
				\Om_1 = \{ S, W\}$ und $\sA_1 = \Pot(\Om_1)$. Die 
				Übergangswahrscheinlichkeiten werden durch ein Baumdiagramm beschrieben.

			\begin{center}
			\begin{tikzpicture}
				[knoten/.style = {circle, draw = black, fill = white, inner sep=0pt,
					minimum size = 0.8cm}, scale = 0.72]
				% Urne mit Bällen
				\draw (0, 0.5) -- ++(0, -1) -- ++(3, 0) -- ++(0, 1);
				\filldraw[black](0.5, 0) circle[radius = 0.3]
												(1.5, 0) circle[radius = 0.3];
				\draw (2.5, 0) circle[radius = 0.3];
				% Baumdiagramm
				\node (p) at ( 6, 	0.0) [minimum size = 0]{};
				\node (s)  at ( 9,  1.2) [knoten]{S}	
					edge [-] node[above, rotate = 21, font=\scriptsize]
						{$P(\{ S\})=\frac{2}{3}$} (p);
				\node (ss) at (13,  2.0) [knoten]{S}	
					edge [-] node[above, rotate = 10, font=\scriptsize]
						{$P(S;\{ S\})=\frac{1}{2}$} (s);
				\node (sw) at (13,  0.8) [knoten]{W}	
					edge [-] node[below,  rotate = 355, font=\scriptsize]
						{$P(S;\{ W\})=\frac{1}{2}$} (s);
				\node (w)  at ( 9, -1.2) [knoten]{W}	
					edge [-] node[below, rotate = 339, font=\scriptsize]
						{$P(\{ W\})=\frac{1}{2}$} (p);
				\node (ws) at (13, -0.8) [knoten]{S}	
					edge [-] node[above, rotate = 5, font=\scriptsize]
						{$P(W;\{ S\})=1$} (w);
				\node (ww) at (13, -2.0) [knoten]{W}	
					edge [-] node[below,  rotate = 350, font=\scriptsize]
						{$P(W;\{ W\})=0$} (w);
			\end{tikzpicture}
			\label{fig: Beispiel 2.3.c}
			\end{center}
	\end{enumerate}

\colorsubsection[Blue]{Definition}
\label{ss: Definition 2.4}
	Sei $(\Om, \sA, P)$ ein W-Raum. Dann heißt $P$
	\begin{enumerate}
		\item \textbf{auf $\bm{A \in \sA}$ konzentriert}, falls $P(A) = 1$ gilt.
		\item \textbf{diskret}, falls $P$ auf einer abzählbaren Menge konzentriert
			ist.
		\item diskrete W-Räume gestatten folgende Charakterisierungen:
	\end{enumerate}

\colorsubsection[Red]{Satz}
\label{ss: Satz 2.5} 
	Sei $(\Om, \sA, P)$ ein W-Raum mit $\{ x\} \in \sA$ für 
	alle $x \in \Om$, dann ist die Teilmenge $\Om \supseteq \Om_0 := \{ \om \in
	\Om\ |\ P(\{ \om \} > 0\}$ äquivalent:
	\begin{enumerate}
		\item $P$ diskret
		\item $P(\Om_0) = 1$
		\item ${ \int_{\Om} f\ dP =  \sum_{x \in \Om_0} 
			f(x) \cdot P(\{	x\})}$ für alle beschränkten und $\sA$-messbaren 
			Abbildungen $f:\Om \rightarrow \Real$
		\item Es gilt $P = \sum_{x \in \Om_0} P(\{ x\}) \cdot \delta_x$\\
			(Dabei ist a) als Gleichheit von Abbildungen $\sA \rightarrow [0, 1]$ 
			gemeint, d.h.: $ \forall A \in \sA: P(A) = \sum_{x 
			\in \Om_0} P(\{ x\}) \cdot \delta_x(A) =  \sum_{x \in A \cap \Om_0} 
			P(\{ x \})$ ). 
	\end{enumerate}

	\textbf{Beweis:} Kommt Später!

	\textbf{Beispiel:} Für das Lebesgue-Maß $\lambda$ auf einem Intervall $[a,b]\ 
		(-\infty < a < b < \infty)$ gilt $\lambda(\{ x\}) = 0$ für alle $x \in \Om =
		[a,b]$. Es folgt $\Om_0 = \emptyset$ und $\lambda(\Om_0) \neq 1$, also ist 
		$\frac{1}{|b-a|} \cdot \lambda$ ein nicht diskretes W-Ma0 auf $[a,b]$

\section{Endliche Wahrscheinlichkeitsräume}
\label{s: 3. Endliche Wahrscheinlichkeitsraeume}
	In diesem Abschnitt betrachten wir W-Räume $(\Om, \sA, P)$ mit $|\Om| < \infty
	$ und $\sA = \Pot(\Om)$. Dann ist P diskret, aber wir nehmen \textbf{nicht}
	an, dass $P\{ x\} = \frac{1}{|\Om|} \forall x \in \Om$ (vgl. 2.3.a).

\colorsubsection[Blue]{Definition}
\label{ss: Definiton 3.1}
	\begin{enumerate}
		\item Für $k, n \in \Nat$ heißt $ \binom{n}{k} := 
				\left \{ \begin{tabular}{cl}
					$ \frac{n!}{k!(n-k)!}$ &, falls $k \leq n$ \\
					$0$ &, sonst
				\end{tabular} \right.$ 
			der \textbf{Binomialkoeffizient \glqq n über k \grqq}.
		\item Für $r > 1;\ k_1, ..., k_r \geq 0$ und $n := k_1 + ... + k_r$ heißt $
			 \binom{n}{k_1, ..., k_r} := \frac{n_1}{(k_1!) \cdot (k_2!)
			\cdot ...  \cdot (k_r!)}$ \textbf{Multinomialkoeffizient}. 

			\textbf{Beispiel:} Für $n,k \geq 0$ gilt $ \binom{n}{k} = 
				\binom{n}{k, n-k}$
	\end{enumerate}

\colorsubsection[Red]{Satz}
\label{ss: Satz 3.2}
	Sei $M$ eine Menge mit $0 \leq |M| =: n \leq \infty$
	\begin{enumerate}
		\item Für $k \geq 0$ besitzt M genau $ \binom{n}{k}$ Teilmengen
			mit $k$ Elementen, d.h. es gilt $$|\ \{ X\ |\ X \subseteq M \textrm{ mit }  
			|X| = k\}\ | = \binom{|M|}{k} = \binom{n}{k}$$
		\item Mit den Bezeichnungen aus 3.1.b ist $(k_1, ..., k_r$ die Anzahl an 
		Möglichkeiten, n paarweise verschiedene Objekte auf r Schachteln der Größen 
		$k_1, ..., k_r$ aufzuteilen.

			\textbf{Beispiel:} Es ist $ \binom{n}{k}$ nach 3.2.a. die 
			Anzahl der k-elementigen Teilmengen von $\{ 1, ..., n\}$, wegen $
			 \binom{n}{k} = \binom{n}{k, n-k}$ nach 3.2.b aber auch die 
			Anzahl der Möglichkeiten, die Objekte $\{ 1, ..., n\}$ auf 2 Schachteln 
			der Größe $k$ und $(n-k)$ zu verteilen.

		\textbf{Beweis:} Kommt später! 
	\end{enumerate}

\colorsubsection[Red]{Satz}
\label{ss: Satz 3.3}
	(Urnenmodell) Aus einer Urne mit $n$ paarweise verschiedenen Kugeln werden 
	k Kugeln nacheinander gezogen. Dann gibt es folgende Anzahl möglicher 
	Ereignisse:

	\begin{tabular}{|c|c|c|}
		\hline
		& mit	zurücklegen & ohne zurücklegen 
		\tabularnewline \hline
		unter Beachtung der Reihenfolge & $\displaystyle n^k$ 
			& $\displaystyle n \cdot (n-1) \cdot ... \cdot (n-k+1)$
		\tabularnewline \hline
		ohne Beachtung der Reihenfolge & $\displaystyle \binom{n+k-1}{k}$ 
			& $\displaystyle \binom{n}{k}$
		\tabularnewline \hline
	\end{tabular}

	\textbf{Beweis:} Kommt später!

\colorsubsection[Red]{Satz}
\label{ss: Satz 3.4}
	Eine urne beinhalte n (paarweise verschiedene) Kugeln, von denen $m$ weiß und
	$n-m$ schwarz seien. Man zieht $l$ Kugeln, und $X \in \{ 0, ..., l \}$ sei die
	Anzahl der gezogenen weißen Kugeln. Bei Annahme eines Laplace Experimentes
	erhält man folgende Wahrscheinlichkeiten für die Werte von $X$.
	\begin{enumerate}
		\item \textbf{Hypergeomerische Verteilung:} Beim Ziehen ohne Zurücklegen
			ist die Wahrscheinlichkeit für $X=k$ gegeben durch $H(l, m, n)\{ k \} :=
			\frac{ \binom{n}{k} \binom{n-m}{l-k} }{ \binom{n}{l} }$ für $0 \leq k \leq 
			\min\{ m, l \}$
		\item \textbf{Bionomialverteilung:} Bei Ziehen mit Zurücklegen ist die 
			Wahrscheinlichkeit für $X=k$ gegeben durch $ B(l, p)\{ k \} :=	
			\binom{l}{k} p^k (1-P)^{l-k} $ für $0 \leq k \leq l$ und $p := 
			\frac{m}{n}$
	\end{enumerate}

	\textbf{Beweis:} Kommt später!

\colorsubsection[Blue]{Definiton}
\label{ss: Definition 3.5}
	Sei $(\Om, \sA, P)$ ein W-Raum (nicht notwendig endlich). Für $A, B \in \sA$
	mit $P(B) \geq 0$ heißt $P(A|B) = \frac{P(A \cap B)}{P(B)}$ die \textbf{
	bedingte Wahrscheinlichkeit von A unter der Bedingung B}.

\colorsubsection[Violet]{Bemerkung}
\label{ss: Bemerkung 3.6}
	\begin{enumerate}
		\item In der Situation von 3.5 ist $(B, \sA \cap B, P(|B|))$ ein W-Raum.
		\item Sei $\Om = \sum_{i \in \Nat,\ P(B_i) > 0} P(B_i)\ P(A|B_i)$ (
		Berechnung einer Wahrscheinlichkeit $P(A)$ durch die Fallunterscheidund $
		\Om = \bigcupdot_{i \in \Nat} B_i$). Fals $A \in \sA$ mit $P(A) \geq
		0$ gegeben ist gilt:
		$$ P(B_i|A) = \frac{P(B_i) \cdot P(A|B_i)}{\sum_{j \in \Nat,\ P(B_j)
		\geq 0}	P(B_j) \cdot P(A|B_j)} \textrm{\hspace{1cm} \textbf{(Formel von 
		Bayes)}}$$
	\end{enumerate}

	\textbf{Beweis:} Kommt später!

\colorsubsection[Violet]{Proposition}
\label{ss: Proposition 3.7}
	\textbf{(Einschluss-Ausschluss-Prinzip)} Es seien $(\Om, \sA, P)$ ein W-Raum
	(nicht unbedingt endlich), $n \geq 1$ und $A_1, ..., A_n \in \sA$. Für eine
	Teilmenge $J \subseteq I := \{ 1, ..., n \}$ setze $B_J := \bigcap_
	{j \in J} A_j \cap (\bigcap_{j \in I \backslash J} A_{j}^C) \in 
	\sA$. Dann gelten:
	\begin{enumerate}
		\item Für jede Teilmenge $J \in I$ gilt $P(\bigcap_{k \in K} A_k)
		= \sum_{K \subseteq J \subseteq K} P(B_J)$
		\item Für jede Teilmenge $J \in I$ gilt $P(B_J) = \sum_{J 
			\subseteq K \subseteq J} (-1)^{|K \backslash J|}\ P(\bigcap_
			{k \in K} A_k)$.\\
			 Inbesondere gilt:
		\item $P(\bigcup_{i \in J} A_i) = 1 - P(B_\emptyset) = \sum 
			_{\emptyset \neq k \subseteq J} (-1)^{|K| + 1}\ P(\bigcap  
			_{k \in K} A_k)$	also zB. \\
		$P(A_1 \cup A_2) = P(A_1) + P(A_2) - P(A_1 \cup A_2) \hspace{1cm} (n=2)$\\
		$P(A_1 \cup A_2 \cup A_3) = P(A_1) + P(A_2) + P(A_3) - P(A_1 \cup A_2) 
			- P(A_1 \cup A_3) - P(A_2 \cup A_3) + P(A_1 \cup A_2 \cup A_3) 
			\hspace{1cm} (n=3)$\\
	\end{enumerate}
	\textbf{Beweis:} Kommt später!

\colorsubsection[Green]{Beispiel}
\label{ss: Beispiel 3.8}
	(Fixpunktfreie Permutationen) Zwei identische Kartenspiele von je 52 Karten 
	werden gemischt und nebeneinander gelegt. Dann werden 52 mal die beiden
	jeweils obersten karten aufgedeckt. Wie hoch ist die Wahrscheinlichkeit
	wenigstens einmal dieselbe Karte aufzudecken?\\
	\textbf{Lösung:} Seien $N := 52,\ I := \{ 1, ..., N \},\ \Om := \sum(I) := 
	\{ \varphi: I \rightarrow I\ |\ \varphi \textrm{ bijektiv} \}$ also $|\Om| =   
	N!$ und $ (\Om, \Pot(\Om), P(\om) = \frac{1}{|\Om|} \forall \om \in \Om)$ der  
	zugehörige Laplace W-Raum (vgl. 23.a~).\\
	Numeriert man die Karten des ersten Stapels als $1, ..., N$ durch, so ist die 
	Reihenfolge des zweiten Stapels durch genau eine Permutation $\om \in \Om$
	festgelegt.
	Da das Mischen fair sein soll, nehmen wir $P(\om) = \frac{1}{|\Om|} \forall
	\om \in \Om$ an. Das fragliche Ereignis ist nun
	$$A := \{ \om \in \Om\ |\ \exists 1 \leq i \leq N: \om(i) = i \} \subseteq
	\Pot(\Om)$$
	Offenbar gilt
	\begin{enumerate}
		\item $ A = \bigcup_{i \in I} A_i $ 
		\item $ P(\bigcap_{k \in K} A_k) = \frac{(N-|K|)!}{N!}$ für jede Teilmenge 
			$K \subseteq I$
	\end{enumerate}

	Nun folgt 
		$$P(A) \mathop{=}\limits^{(a)} \sum_{\emptyset \neq K \subseteq I} (-1)^{|K| 
			+1} \cdot P(\bigcap_{k \in K} A_k) \mathop{=}\limits^{(b)} \sum_{\emptyset 
			\neq K \subseteq I} (-1)^{|K|+1} \cdot \frac{ (N-|K|)! }{N!} $$
	(Summanden hängen von $K$ nur über $|K|$ ab.)

	\begin{align*}
		&\sum_{k=1}^N (-1)^{|K|+1} \cdot \frac{ (N-|K|)! }{N!} \cdot |\ \{ k\ |\ 
			\emptyset \neq K \subseteq I: |K| = k\}\ | = \\ 
		= &\sum_{k = 1}^N (-1)^{k+1} \cdot \frac{(N-k)}{N!} \cdot \binom{N}{k} = \\
		= &\sum_{k=1}^N (-1)^{k+1} \cdot \frac{1}{k!} = \\
		= &1 - e^{-1} - \underbrace{\sum_{k \geq N + 1}	(-1)^{k+1} \cdot \frac{1}
			{k!}}_{R(N)}
	\end{align*}

	Für $N = 52$ ist $|R(N)| \leq \frac{1}{N!} \leq 10^{-60}$, also $P(A) \approx
	10^{-60}$, also $P(A) \approx 1 - e^{-1} \approx$ 0.63 

	\textbf{Tabelle für kleines N: }
	\begin{tabular}{c|c|c|c|c}
		N & 1 & 2 & 3 & 4 
		\tabularnewline \hline
		$P(A)$ &\rule{0pt}{3ex} 1 & $\frac{1}{2}$ & $\frac{2}{3}$ & $\frac{3}{8}$ 
	\end{tabular}

	\textbf{Fixpunkte für $\bm{\om \in \Om = \sum(\{ 1,2,3 \})}$:}

	\begin{tabular}{c|c|c|c|c|c|c}
			& 1 & 1 & 2 & 2 & 3 & 3 \\
			& 2 & 3 & 1 & 4 & 1 & 1 \\
			& 3 & 2 & 3 & 1 & 2 & 1 
			\tabularnewline \hline
		Fixpunkte & 3 & 1 & 1 & 0 & 0 & 1
	\end{tabular}
	\hspace{1cm} $\displaystyle \rightsquigarrow \hspace{1cm}\frac{4}{6} = 
		\frac{2}{3} \approx 0.66$

\section{Zufallsvariablen, Erwartungswert und Varianz}
\label{s: 4. zufallsvariablen, Erwartungswert und Varianz}

\colorsubsection[Blue]{Definition}
\label{ss: Definition 4.1}
	\begin{enumerate}
		\item Seien $\Om$ eine Menge und $(X_i)_{i \in I}$ eine Familie von 
			Abbildungen $X_i:\Om \rightarrow S_i$, wobei $(S_i, \sS_i)$ ein Messraum
			ist. Dann heißt $\sigma((X_i)_{i \in I} := \sigma(\bigcup_{i \in I} 
			X_i^{-1} (\sS_i)) \subseteq \Pot(\Om)$ \textbf{die von $\bm{(X_i)_{i  
			\in I}}$ erzeugte $\bm{\sigma}$-Algebra.}
		\item Seien $(\Om, \sA)$ und $(S,\sS)$ W-Räume. Dann heißt eine Abbildung 
			$X: \Om \rightarrow S$ eine \textbf{Zufallsfunktion}, falls $X:\sA 
			\rightarrow \sS$ messbar ist. D.h. wenn für alle $S \in \sS$ gilt: $X^{-1}
			(\sS) \in \sA$. Das Bildmaß $P \circ X^{-1}: \sS \rightarrow [0,1], B 
			\rightarrow  P(X^{-1}(B)) =: P(\{ X \in B \})$ heißt die \textbf{
			Verteilung oder W-Verteilung von X.}
		\item Nun seien speziell $(S, \sS) = (\Real, \Bor(\Real))$ und $X:(\Om, \sA)
			\rightarrow (\Real, \Bor(\Real))$ messbar. Dann heißt X \textbf{
			Zufallsvariable (ZV)}  und die Abblildung $F_X:\Real \rightarrow [0,1], 
			F_X(t) := P(X^{-1}(-\infty, t]) = P\{ (X \leq t \}$ heißt die 
			\textbf{Verteilungsfunktion von X.}
	\end{enumerate}

	Die ZV X heißt \textbf{diskret verteilt}, falls $P \circ X^{-1}$ diskret ist
	und \textbf{stetig verteilt}, falls $P \circ X^{-1}$ absolutstetig bzgl. des 
	Lebesgue-Maßes ist. D.h. eine Dichte $f_X$ exisitiert mit $P \circ X^{-1} = 
	f_X(t)\ d\lambda$ (Gleichheit von Maßen auf $(\Real, \Bor(\Real))$).

\colorsubsection[Violet]{Bemerkung}
\label{ss: Bemerkung 4.2}
	\begin{enumerate}
		\item $F_X$ ist monoton wachsend, rechtsstetig (r.c.), d.h. es gilt 
			$\lim_{t \rightarrow -\infty} F_X(t+h) = F_X(t)\ \forall\ t \in \Real$, 
			und es gelten $\lim_{t \rightarrow -\infty} F_X(t) = 0$ und $\lim_{t 
			\rightarrow + \infty} F_X(t) = 1$. Außerdem gilt für alle $t \in \Real$: 
			$$ F_X(t) - \lim_{h > 0, h \rightarrow 0} F_X(t-h) = (P \circ X^{-1}(\{ 
			t\})$$, 
			also sind die Sprungstellen von $F_X$ genau diejenigen $t \in \Real$ mit
			$P({X = t}) \neq 0$. 
		\item $X$ ist genau dann stetig verteilt, wenn $F_X$ absolut stetig ist,
			d.h. wenn $F_X(t) = \int_{-\infty}^t F_X(s)\ \textrm{ds}$ mit einer  
			messbaren Funktion $f_X: \Real \rightarrow \Real_f$ und $\int_\Real f_X(s)
			\textrm{ds} = 1$ gilt(zB. wenn $F_X$ stetig differenzierbar ist).
	\end{enumerate}

\colorsubsection[Red]{Satz}
\label{ss: Satz 4.3}
	Sei $(\Om, \sA, P)$ ein W-Raum.
	\begin{enumerate}
		\item Sind $X$ und $Y$ Zufallsvariablen auf $\Om$, so auch $\max\{ X, y \},\
			\min\{ X, y \},\ |X|$ für $r > 0$ und $X^r$ für $r \in \Nat$	
		\item Ist ${(X_n)}_{n \in \Nat}$ eine Folge von Zufallsvariablen auf $\Om$, 
			so sind $\inf_n X_n$, $\sup_n X_n$, $\liminf_n X_n$ und $\limsup_n X_n$ 
			Zufallsfunktionen von $ \Om$ nach $(\overline{\Nat}, \overline{\Real}$.

			\textbf{Beweis:} Kommt später!
	\end{enumerate}

\colorsubsection[Red]{Satz}
\label{ss: Satz 4.4}
	(Fouriertransformation) Seien $\sM$ die Menge aller W-Maße $(\Real, \Bor(
	\Real))$ und $\sF$ die Menge aller Abbildungen von $\Real$ nach $\Real$,
	welche monoton wachsen, r.c. sind und gegen 0 (bzw. 1) konvergieren für $t 
	\rightarrow -\infty$ (bzw. $+ \infty$). Dann ist die Abbildung
	$$ \sM \rightarrow \sF,\ \mu \rightarrow F_\mu \textrm{ mit } F_\mu(t) := 
	\mu( (-1,t])$$ eine Bijektion.

\colorsubsection[Blue]{Definition}
\label{ss: Defintion 4.5}
	Zwei ZV $X,Y: \rightarrow (\Om, \sA) \rightarrow (\Real, \Bor(\Real))$ heißen 
	\textbf{stochastisch äquivalent} (geschrieben $X \sim Y$), falls $P \circ 
	X^{-1} = P \circ Y^{-1}$.

\colorsubsection[Green]{Beispiel}
\label{ss: Beispiel 4.6}
	Für $X \sim Y$ folgt nicht $X = Y$, und noch nicht mal P({X = Y}) = 1, wie 
	wie folgendes Beispiel zeigt: Es sei $(\Om, \sA, P) := (\{ 0,1 \}, \Pot(\{ 0,1
	\}), \frac{1}{2}(\delta_0 + \delta_1))$. Setzt man $X(t) := t$ und $Y(t) := 
	1 - t$ für $t \in \Om$, so ist $P \circ X^{-1} = P \circ Y^{-1} = 1/2(\delta_0
	+ \delta_1)$, also gilt $X(t) \neq Y(t) \forall t \in \Om$ und damit $P\{ X =
	Y \} = 0$.

\colorsubsection[Red]{Satz}
\label{ss: Satz 4.7}
	(Erinnerungen an das Lebesgue-Integral) Seien X und Y Zufallsvariablen auf dem
	W-Raum $(\Om, \sA, P)$.
	\begin{enumerate}
		\item $X$ ist genau dann integrierbar, wenn $\int_\Om |X| dP < \infty$
		\item (Linearität) Seien $X,Y$ integrierbar auf $a,b \in \Real$. Dann gilt
			$\int_\Om (aX + bY) dP = a \cdot \int_\Om X dP + b \cdot \int_\Om Y dP$
		\item Seien $X$ integrierbar, ${(A_n)}_{n \in \Nat}$ paarweise disjunkt und
			$A := \bigcup_{n \geq 1} A_n$. Dann ist $\int_A X dP = \sum_{n \geq 1} X
			dP$
		\item (Positivität) Aus $X \geq 0$ \textbf{P-fast sicher} (P-f.s., d.h. $
			P\{ \om \in \Om\ |\ X(\om) \leq 0 \}$) folgt $\int_\Om X dP \geq 0$
		\item (Monotonie) Seien $X, Y$ integrierbar mit $X \leq Y P-f.s.$. Dann gilt
			$\int_\Om X dP \leq \int_\Om Y dP$
		\item Seien $X$ integrierbar und $a,b \in \Real$ ist $a \leq X \leq b$ 
			P-f.s. auf einer Menge $A \in \sA$. Dann gilt $a \cdot P(A) \leq \int_A
			X dP \leq b \cdot P(A)$
		\item (Dreiecksungleichung) Für $X$ integrierbar gilt $| \int_\Om X dP |
			\leq \int_\Om |X| dP$ 
		\item (Satz von majorisierter Konvergenz) Seien $X_n, X$ und $Y$ 
			Zufallsvariablen mit 
			\begin{enumerate}
				\item ${|X_n| \leq Y}$ P-f.s. $\forall n \in \Nat$
				\item $\int_\Om Y	dP < \infty$
				\item $X_n \rightarrow X$ für $n \rightarrow \infty$ P-f.s.
			\end{enumerate}  
			Dann gilt: $\lim_{n \rightarrow \infty} \int_\Om X_n dP = \int_\Om X dP$
		\item (gliedweise Integration) Seien $X_n$ Zufallsvariablen mit $\sum_n 
			(\int_\Om |X_n| dP) \leq \infty$. Dann gilt $\int_\Om (\sum_n |X_n|) dp 
			\leq  \infty$ P-f.s. und es gilt $\int_\Om (\sum_n X_n) dP = \sum_n (\int
			_\Om X_n dp)$
		\item (Satz von majorisierbarer Konvergenz) Seien $X_n \geq 0$
			Zufallsvariablen und $X$ eine Zufallsvariable mit $X_n \nearrow X$ für $
			n \rightarrow \infty$ P-f.s.. Dann gilt $\lim_{n \rightarrow \infty}
			\int_\Om X_n dP = \int_\Om X dP$ (wobei beide Seiten $+ \infty$ sein 
			können)
		\item (Lemma von Fatou) Seien $X_n \geq 0$ Zufallsvariablen. Dann ist
			$$ \int_\Om \liminf_{n \rightarrow \infty} X_n dP \leq \liminf_{n 
			\rightarrow \infty} \int_\Om X_n dP$$
	\end{enumerate}

\colorsubsection[Blue]{Definition}
\label{ss: Defintion 4.8}
	Sei $X$ eine Zufallsvariable auf dem W-Raum $(\Om, \sA, P)$
	\begin{enumerate}
		\item Falls $X \in L^1(P) := \{ X | \int_\Om |X| dP \leq \infty \}$ gilt, so
			heißt $\E X := \int_\Om X dP$ der \textbf{Erwartungswert von X} (auch 
			sinnvoll falls X mit möglichem Wert $+\infty$).
		\item Falls $X \in L^2(P) := \{ X| \int_\Om |X^2| dP < \infty \}$ (und 
			damit auch $X \in L^1(P)$) gilt, so heißt $\Var X := \E[{(X - \E X)}^2] 
			= \int_\Om (X - \E X)^2 dP \geq 0$ die \textbf{Varianz von X}. Die Zahl $
			\sigma X := +\sqrt{\Var(X)}$ heißt \textbf{die Streuung von X}.
		\item Für $n \geq 1$ heißt $\mu_n := \E(X^n)$ \textbf{das n-te Moment von
			X} (falls es existiert).
	\end{enumerate}
	
	\textbf{Bemerkung:}
	\begin{enumerate}
		\item Die Eigenschaften des Integrals übertragen sich auf den 
			Erwartungswert, zB. existiert $\E X$ genau dann, wen $\E X \leq \infty
			$ gilt.
		\item Es gilt $\E 1 = 1$ 
		\item Nach dem Transformationslemma gilt $X \in L^1(P)$ genau dann, wenn $
			id_\Real \in L^1(P \circ X^{-1})$. Die Existenz und der Wert von $\E X$ 
			und $\Var X$ hängen nur von $P \circ X^{-1}$ ab, nämlich:
			$$ \E X = \int_\Om X dP = \int_\Real (P \circ X^{-1}) dt \textrm{ für }
			X \in L^1(P)$$
		\item Allgemein in c) gilt für $g: \Real \rightarrow \Real$ mit $g \circ X 
			\in L^1(P)$ (zB. $g: \Real \rightarrow \Com$)

			$$ \int_\Om (g \circ X) dP = \int_\Real g\ d(P \circ X^{-1}) =
			\left \{\begin{tabular}{cl}
					$\displaystyle \sum_{x \in \Real} g(x)  \cdot P\{X = x\}$ &, 
						falls $X$ diskret\\
					$\displaystyle \int_\Real gf_x d\lambda$ &, 
						falls $X$ stetig
			\end{tabular} \right. $$

			$$ \E X = \left \{\begin{tabular}{cl}
					$\displaystyle \sum_{x \in \Real} x \cdot P\{X = x\}$ &, 
						falls $X$ diskret\\
					$\displaystyle \int_\Real f_x d\lambda$ &, 
						falls $X$ stetig verteilt ist mit Dichte $f_x$
			\end{tabular} \right. $$
	\end{enumerate}

\colorsubsection[Red]{Satz}
\label{ss: Satz 4.9}
	Sei X eine Zufallsvariable auf dem W-Raum $(\Om, \sA, P)$ und $f:\Real
	\rightarrow \Com$ \textbf{stetig} mit $f \circ X \subseteq L^1(P)$. Dann gilt
	$\int_\Om (f \circ X) dP = \int_{-\infty}^{\infty} f(t)\ dF_X(t)$, wobei auf
	der rechten Seite das Riemann-Integral steht (und $F_X$ die Dichtefunktion
	von X ist.)

	\textbf{Beweis:} Kommt später!

\colorsubsection[Red]{Satz}
\label{ss: Satz 4.10}
	(Diskrete Approximation des Erwartungswerts) Sei X eine Zufallsvariable auf
	dem W-Raum$(\Om, \sA, P)$. Dann gilt $\E |X| \leq 1 + \sum_{n \geq 1} P\{ 
	|X| \geq n \}$.

	\textbf{Beweis:} Kommt später!

\colorsubsection[Orange]{Lemma}
\label{ss: Lemma 4.11}
	Seien $(\Om, \sA, P)$ ein W-Raum und $X: \Om \rightarrow \Real$ eine ZV.
	\begin{enumerate}
		\item Für $X \in L^2(P)$ ist $\Var X = E(X^2) - (EX)^2$
		\item Für $X \in L^2(P)$ und $\alpha, \beta \in \Real$ gilt $\Var(\alpha X
			+ \beta) = \alpha^2 \cdot \Var(X)$
	\end{enumerate}

	\textbf{Beweis:} Kommt später!

\colorsubsection[Red]{Satz}
\label{ss: Satz 4.12}
	(Ungleichung von Chebychev) Seien $(\Om, \sA, P)$ ein W-Raum, $X: \Om 
	\rightarrow \Real$ eine ZV, $1 \leq p < \infty$ und es gelte $X \in L^p(P)$ 
	(d.h. $||X||_{L^p}^p := \int_\Om |X|^p dP < \infty$). Dann gilt für alle $c 
	> 0$ 
	$$P\{ |X| \geq c \} \leq \frac{1}{c^p} \cdot ||X||_{L^p}^p$$ 
	Insbesondere gilt für $X \in L^2(P)$
	$$ P\{ |X - \E X| \geq c \} \leq \frac{1}{c^2} \cdot \Var (X)$$
	(daher der Begriff Varianz).

	\textbf{Beweis:} Kommt später!

\colorsubsection[Blue]{Definition}
\label{ss: Definition 4.13}
	Seien $(\Om, \sA, P)$ ein W-Raum und $X, Y: \Om \rightarrow \Real$ ZV. Es 
	gelte $X, Y, XY \in L^1(P)$ (zB. $X, Y \in L^2(P)$, vgl. Hölderungleichung).
	Dann heißt $$\Cov(X, Y) := \E [(X - \E X) (Y - \E Y)]$$ \textbf{die Kovarianz
	von $X$ und $Y$}. Die ZV $X, Y$ heißen \textbf{unkorreliert}, falls $\Cov(X,Y)
	= 0 $ gilt. Gilt zusätzlich $\Var X, \Var Y > 0$, so heißt $$\rho(X,Y) :=
	\frac{\Cov(X,Y)}{(\Var X)^{\frac{1}{2}} (\Var Y)^{\frac{1}{2}}}$$ der \textbf{
	Korrelationskoeffizient von $X$ and $Y$}.

\colorsubsection[Red]{Satz}
\label{ss: Satz 4.14}
	Seien $(\Om, \sA, P)$ ein W-Raum und $X_1, ..., X_n:\Om \rightarrow \Real$ ZV
	mit $X_1, ..., X_n \in L^2(P)$. Dann gilt 
	$$\Var (\sum_{i = 1}^n X_i) = \sum_{i=1}^n \Var X_i + \sum_{i,j=1;\ i \neq j}
	^n \Cov(X_i, X_j)$$
	Falls $X_1, ... X_n$ paarweise unkorreliert sind, so folgt 
	$$\Var( \sum_{i=1}^n X_i) = \sum_{i=1}^n \Var X_i \hspace{1 cm}
	\textrm{(Formel von Bienayme)}$$

	\textbf{Beweis:} Kommt später!

\section{Beispiele wichtiger Wahrscheinlichkeitsverteilungen}
	Ist $X: (\Om, \sA, P) \rightarrow (\Real, \Bor(\Real))$ eine ZV, so heißt das 
	Maß $P \circ X^{-1}$ auf $(\Real, \Bor(\Real))$ die Verteilung von X. Viele 
	Fragen hängen vonn X nur über $P \circ X^{-1}$ (und zB. nicht von $(\Om, \sA, 
	P)$) ab. Deswegen wird im Folgenden nur das Maß $P \circ X^{-1}$ angegeben 
	(obwohl weder P noch X definiert sind). 

\colorsubsection[Green]{Beispiel}
\label{ss: Beispiel 5.1}
	\begin{enumerate}
		\item \textbf{Die Gleichverteilung auf $\bm{[a,b]}$} mit $-\infty < a < b <
			\infty$ ist 
			$$P \circ X^{-1} := \frac{1}{|b-a|} \cdot 1_{[a,b]} \cdot \lambda$$
		\item \textbf{Die Binomialverteilung zu den Parametern $\bm{n \in \Nat_{
			\geq 1}}$	und $\bm{p \in [0,1]}$} ist definiert als 
			$$ P \circ X^{-1} := B(n;p) := \sum_{k=0}^n \binom{n}{k} p^k (1-p)^{n-k} 
			\delta_k$$
			Es gilt $\E(B(n;p)) = np$ und $\Var(B(n;p)) = np(1-p)$.
		\item	f
		\textbf{Die Poissonverteilung zum Parameter $\bm{\lambda > 0}$} ist 
			definiert als 
			$$P \circ X := {\Pi_\lambda|}_{\Bor(\Real)}$$
			wobei $ \Pi_\lambda: \Pot(\Real) \rightarrow [0,1] $ gegeben ist als 
			$$\Pi_\lambda := \sum_{k=0}^\infty e^{-\lambda} \cdot \frac{\lambda^k}{k!}
				\cdot \delta_k$$
			Es ist $X$ diskret verteilt und auf $\Nat_0$ konzentriert. Es gilt $\E X
			= \Var X = \lambda$. Für $\E X$ sieht man das folgendermaßen: 

			\begin{align*}
				\E X &= \int_\Real id\ d(P \circ X^{-1}) 
					= \sum_{k=0}^\infty k \cdot P\{ X = k \} 
					= \sum_{K=0}^\infty k \cdot e^{-\lambda} \cdot \frac{\lambda^k}{k!} 
					= \\
					&= e^{-\lambda} \sum_{k=1}^\infty \frac{\lambda^k}{k-1} 
					= e^{-\lambda} \cdot \lambda \sum_{k=0}^\infty \frac{\lambda^k}{k!}
					= e^{-\lambda}\cdot \lambda \cdot e^\lambda 
					= \lambda
			\end{align*}
		
		\item \textbf{Die Exponentialverteilung zum Parameter $\bm{\lambda > 0}$ ist
			definiert als}
			$$ P \circ X^{-1} = f_\lambda (t) := 1_{\Real^{+}} (t) \cdot \lambda 
				e^{-\lambda t}$$ 
			(d.h. $X$ ist stetig verteilt). Für die Verteilungsfunktion erhält man
			$$ F_X(t) := \int_{-a}^t f_x(s) ds = \left\{ \begin{tabular}{cl}
				0, & $t < 0$\\
				$1 - e^{-\lambda t}$, & $t \geq 0$
				\end{tabular}
			\right. $$
			Es gelten $\E X = \frac{1}{\lambda}$ und $\Var X = \frac{¹}{\lambda^2}$.


 	\end{enumerate}

\colorsubsection[Blue]{Definition}
\colorsubsection[Blue]{Definition}
\colorsubsection[Blue]{Definition}
\colorsubsection[Blue]{Definition}
\colorsubsection[Blue]{Definition}
\colorsubsection[Blue]{Definition}

\section{Kapitel}
\colorsubsection[Blue]{Definition}
\colorsubsection[Blue]{Definition}
\colorsubsection[Blue]{Definition}
\colorsubsection[Blue]{Definition}
\colorsubsection[Blue]{Definition}
\colorsubsection[Blue]{Definition}
\section{Kapitel}
\colorsubsection[Blue]{Definition}
\colorsubsection[Blue]{Definition}
\colorsubsection[Blue]{Definition}
\colorsubsection[Blue]{Definition}
\colorsubsection[Blue]{Definition}
\colorsubsection[Blue]{Definition}

\colorsubsection[Blue]{Definition}
\label{ss: Defintion 7.7}
	Sei $(\Om, \sA, P)$ ein W-Raum und $X_n: \Om \rightarrow \Real $ ZV 
	$(n \geq 1)$ mit $X_n \in L^1(P)\ \forall n \geq 1$. Dann \textbf{genügt 
	$\bm{{(X_n)}_n}$ dem Gesetz der großen Zahlen} 
	$:\Longleftrightarrow$ Für $S_n := X_1 + ... + X_n$ gilt $\frac{S_n - \E S_n}
	{n} \rightarrow 0$ stochastisch, d.h. wenn gilt
	$$\forall \epsilon > 0, \delta > 0\ \exists N = N(\epsilon, \delta): \forall
	n \geq N: {P\left\{ \frac{|S_n - \E S_n|}{n} > \epsilon \right\} < \delta}$$

	\textbf{Bemerkung:} Seien $X_1, X_2, ...$ integrierbare ZV und $\forall i 
		\geq 1$ gelte $\E X_i = \E X_1 =: \E$. Dann gilt 
		$$\frac{S_n - \E S_n}{n} =\left( \frac{1}{n} \sum_{i=1}^n X_i\right) - \E$$
		d.h. man betrachtet in der Def. 7.7 die Differenz zwischen dem Mittelwert 
		(\glqq rel. Häfuigkeit \grqq) und dem Erwartungswert.

\colorsubsection[Green]{Beispiel}
\label{ss: Beispiel 7.8}
	Sei ${(X_n)}_{n \geq1}$ eine Folge unabhängiger ZV, die alle Bernoulliverteilt
	sind mit Parameter $p = \frac{1}{2}$ (zB. wiederholter Münzwurf einer fairen
	Münze). Wir sehen in 7.9, dass ${(X_n)}_{n \geq 1}$ dem schwachen Geesetz der
	großen Zahlen genügt, d.h. es gilt 
	$$ \forall \epsilon > 0, \delta > 0\ \exists N = N(\epsilon, \delta): \forall
	n \geq N: P\left\{ \left| \frac{1}{n} \sum_{i=1}^n X_i - \frac{1}{2}  
	\right| > \epsilon \right\} < \delta$$
	d.h. die relative Häufigkeit $\frac{1}{n} \sum_{i=1}^n X_i$ nähert sich dem 
	Erwartungswert $\frac{1}{2}$ beliebig genau an.

	Was ist aber mit den Ereignissen $A_n := \left\{ \frac{1}{n} \sum_{i=1}^n X_i 
	= \frac{1}{2} \right \}$, d.h. das \textbf{exakte} Übereinstimmen von rel. 
	Häufigkeit und Erwartungswert? Offenbar gilt $P(A_n) \mathop{\longrightarrow}
	\limits^{n \rightarrow \infty} 0$. Wir betrachten dazu 

	\textbf{Satz (Stirling-Formel)}
	$$\forall n \geq 0 \textrm{ gilt } n! = \sqrt{2\pi n} \cdot 
	n^n e^{-n + \eta(n)} \textrm{ für }	0 < \eta(n) < \frac{1}{12n}$$
	Insbesondere gilt: 
	$$\lim_{n \rightarrow \infty} \sqrt{2\pi n} \cdot n^n e^{-n} \cdot 
	\frac{1}{n!} = \lim_{n \rightarrow \infty} e^{-\eta(n)} = 1 \hspace{1cm}
	\textrm{ (ohne Beweis)}$$ 

	Damit berechnen wir für $n \geq 1$
	$$ P\left( \frac{1}{2n} \sum_{k=1}^{2n} X_k = \frac{1}{2}\right) 
	=	B(2n; \frac{1}{2})(\{ n\}) 
	= \binom{2n}{n} {\left(\frac{1}{2}\right)}^n {\left(1- \frac{1}{2}\right)}
		^{2n-n}  
	= \frac{(2n)!}{(n!)^2} \cdot 2^{-2n}$$
	und mit Stirling-Formel gilt
	$$\lim_{n \rightarrow \infty} \frac{(2n)!}{(n!)^2} \cdot 2^{-2n} 
	= \lim_{n \rightarrow \infty} 
		\frac{ \sqrt{2\pi 2n} \cdot {(2n)}^{2n} \cdot e^{-2n + \eta(2n)} }
			{2\pi n \cdot n^{2n} e^{-2n + 2\eta(n)}}
		\cdot 2^{-2n}
	= \lim_{n \rightarrow \infty} \frac{1}{\sqrt{\pi n}} = 0	$$

\colorsubsection[Red]{Satz}
\label{ss: Satz 7.9}
	\textbf{(Schwaches Gesetz der großen Zahlen)} Sei $(\Om, \sA, P)$ ein
	W-Raum, $X_n : \Om \rightarrow \Real\ (n \geq 1)$ Zufallsvariablen, ${(X_n)}_
	{n \geq 1}$	paarweise	unkorreliert (zB. paarweise unabhängig), $X_n \in L^2(P)
	\ \forall n \geq 1$	und $v := \sup_{n \geq 1} \Var(X_n) < \infty$. Dann gilt
	$$ \forall \epsilon > 0: P\left( \left| \frac{1}{2} \sum_{i=1}^n (X_i 
		- \E X_i ) \right| \geq \epsilon \right) \leq \frac{v}{n \epsilon^2} 
		\hspace{1cm} \textrm{($\star$)}$$
	Insbesondere gilt 
	$$\frac{1}{2} \sum_{i=1}^n (X_i	- \E X_i) \rightarrow 0 \textrm{ stochastisch}
	$$
	d.h. ${(X_n)}_{n \geq 1}$ genügt dem schwachen Gesetz der großen Zahlen. Gilt
	zusätzlich 
	$$\E X_n = \E X_1\ \forall n \geq 1$$
	so gilt 
	$$\frac{1}{n} \sum_{i = 1}^n X_i \rightarrow \E X_1 \textrm{ stochastisch}$$
	\textbf{Beweis:} Kommt später!

\colorsubsection[Green]{Beispiel}
\label{ss: Beispiel 7.10}
	In einem Gefäß befinden sich $n = \frac{1}{4} \cdot 10^{23}$ Gasmoleküle, die
	sich jeweils unabhängig voneinander mit gleicher Wahrscheinlichkeit in der 
	linken oder rechten Seite des Gefäßes befinden. Wie hoch ist die 
	Wahrscheinlichkeit, dass sich in der rechten Kammer mindestens $\frac
	{1 + 10^{-8}}{2}$ aller Teilchen befinden. Seien $X_1, ..., X_n$ unabhängige,
	zum Parameter $p = \frac{1}{2}$ Binomialverteilte ZV ($X_i = 0$ heißt \glqq
	i-tes Teilchen in der linken Hälfte\grqq).
	\begin{align*}
		P\left( \frac{1}{n} \sum_{i=1}^n X_i \geq \frac{1+10^8}{2}  \right) 
			& \symtext{=}{Symmetrie}
		\frac{1}{2} P \left( \left| \frac{1}{n} \sum_{i=1}^n \left( X_i - 
			\frac{1}{2} \right)	\right| \geq \frac{10^{-8}}{2} \right) 
			\symtext{\leq}{($\star$) in 7.9}
		\frac{1}{2} \cdot \frac{\Var X_1}{n(\frac{1}{2} \cdot 10^{-8})} \\
			& \symtext{=}{$\Var X_1 = \frac{1}{4}$}
		\frac{1}{2} \cdot \frac{1}{4n \cdot 10^{-16}} 
			\symtext{=}{$n = \frac{1}{4} \cdot 10^{23}$}
		2 \cdot 10^{-7}
	\end{align*}
	(Man kann sogar zeigen, dass $P\left( \frac{1}{n} \sum_{i=1}^n X_i \geq 
	\frac{1+10^8}{2}  \right) < 10^{-50000}$. Tschebychev sehr grobe
	Abschätzung!)

\colorsubsection[Blue]{Defintion und Satz}
\label{ss: Definition und Satz 7.11}
	Seien $n \geq 1, \mu_1, ..., \mu_n$ endliche Maße auf $(\Real, \Bor(\Real))$ 
	(z.B. W-Maße) und $A_n: \Real^n \rightarrow \Real$ definiert durch $A_n
	((x_1, ..., x_n)) := x_1 + ... + x_n$. Dann heißt das Maß 
	$$\mu_1 * ... * \mu_n	:= (\mu_1 \otimes ... \otimes \mu_n) \circ A_n^{-1} 
	\textrm{ auf } (\Real, \Bor(\Real)) $$ 
	\textbf{das Faltungsprodukt der Maße $\mu_1, ..., \mu_n$}.\\
	Gilt $\mu_i = P \circ X_i^{-1}$ für unabhängige Zufallsvariablen $X_1, ..., 
	X_n$, so folgt 
	$$ P \circ {(\sum_{i=1}^n X_i)}^{-1} = \mu_1 * ... * \mu_n = 
		(P \circ X_1^{-1}) * ... * (P \circ X_n^{-1})$$

	\textbf{Beweis:} Kommt später!


	\end{document}

